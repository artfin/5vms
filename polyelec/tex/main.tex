\documentclass[12pt]{article}

\usepackage[T1]{fontenc}
\usepackage[utf8]{inputenc}
\usepackage[russian]{babel}

\usepackage{amsmath}
\usepackage{float}
\usepackage{tabularx}

% page margin
\usepackage[top=2cm, bottom=2cm, left=2cm, right=2cm]{geometry}

\usepackage{graphicx}

\begin{document}

\begin{figure}
	\includegraphics[width = \linewidth]{../titr.png}
	\caption{Кривая титрования пропионовой кислоты (красный) и ПАК (синий)}
\end{figure}

\begin{figure}
	\includegraphics[width = \linewidth]{../pk.png}
	\caption{Зависимости $pK$ от $\alpha$ пропионовой кислоты (красный) и ПАК (синий)}
\end{figure}

\begin{figure}
	\includegraphics[width = \linewidth]{../eta.png}
	\caption{Зависимость удельной вязкости раствора ПАК от $\alpha$}
\end{figure}

\end{document}

